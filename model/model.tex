\documentclass{article}
\usepackage{amsmath}
\usepackage{amssymb}
\usepackage{geometry}
\geometry{a4paper, margin=1in}
\usepackage{hyperref}

\title{Mathematical Formulation for the 7-Day Task Scheduler}
\author{Based on Python PuLP Model}
\date{\today}

\begin{document}

\maketitle

\section{Problem Definition}
The goal is to schedule a set of tasks over a 7-day horizon, divided into discrete 15-minute time slots. The schedule should respect task deadlines, avoid overlapping tasks, avoid pre-defined blocked time intervals (commitments), adhere to user-defined daily scheduling windows and task-specific time-of-day preferences, while maximizing available leisure time and minimizing a measure of "stress" associated with scheduled tasks (based on priority and difficulty).

\section{Parameters and Sets}

\subsection{Sets}
\begin{itemize}
    \item $I = \{1, \dots, n\}$: Set of tasks to be scheduled.
    \item $T = \{0, \dots, S-1\}$: Set of discrete time slots over the 7-day horizon. $S = \text{TOTAL\_SLOTS} = 392$.
    \item $T_{day} = \{0, \dots, \text{SLOTS\_PER\_DAY}-1\}$: Set of slot indices within a single day (0 to 55).
    \item $D = \{0, \dots, \text{TOTAL\_DAYS}-1\}$: Set of days (0 to 6).
\end{itemize}

\subsection{Task Parameters (for each task $i \in I$)}
\begin{itemize}
    \item $p_i$: Priority of task $i$ (integer, e.g., 1-5).
    \item $d_i$: Difficulty of task $i$ (integer, e.g., 1-5).
    \item $dur_i$: Duration of task $i$ in number of time slots.
    \item $dl_i$: Deadline slot index for task $i$. The task must be completed (i.e., its last slot must end) at or before slot $dl_i$.
    \item $Pref_i \subseteq T$: Set of allowed starting slots for task $i$ based on user preference (e.g., morning, afternoon, evening slots).
\end{itemize}

\subsection{Time Slot Parameters (for each slot $s \in T$)}
\begin{itemize}
    \item $commit_s$: Duration (in minutes, here 15) of fixed commitment/blocked time in slot $s$. $commit_s = 15$ if slot $s$ is blocked, $commit_s = 0$ otherwise.
\end{itemize}

\subsection{Scheduling Window Parameters}
\begin{itemize}
    \item $h_{start}$: Start hour of the daily scheduling window (e.g., 8 for 8:00).
    \item $h_{end}$: End hour of the daily scheduling window (e.g., 17 for 17:00).
    \item $s_{offset}^{start} = (h_{start} - 8) \times 4$: Starting slot index offset within any day.
    \item $s_{offset}^{end} = (h_{end} - 8) \times 4$: Ending slot index offset (exclusive) within any day.
\end{itemize}

\subsection{Objective Function Parameters}
\begin{itemize}
    \item $\alpha$: Weight for maximizing total leisure time (typically 1.0).
    \item $\beta$: Weight for minimizing total stress (typically 1.0).
\end{itemize}

\section{Decision Variables}
\begin{itemize}
    \item $X_{i,s} \in \{0, 1\}$: Binary variable. $X_{i,s} = 1$ if task $i$ starts at time slot $s$, and $0$ otherwise. ($\forall i \in I, s \in T$)
    \item $Y_s \in \{0, 1\}$: Binary variable. $Y_s = 1$ if time slot $s$ is occupied by any task, and $0$ otherwise. ($\forall s \in T$)
    \item $L_s \in \mathbb{R}_{\ge 0}$: Continuous variable representing the amount of leisure time (in minutes) available in time slot $s$. ($\forall s \in T$)
\end{itemize}

\section{Objective Function}
The objective is to maximize the total leisure time minus the total "stress", where stress for a task is its priority multiplied by its difficulty.
\[
\text{Maximize} \quad Z = \alpha \sum_{s \in T} L_s - \beta \sum_{i \in I} \sum_{s \in T} X_{i,s} (p_i \cdot d_i)
\]
Note: The stress term $\sum_{s \in T} X_{i,s} (p_i \cdot d_i)$ simplifies to just $p_i \cdot d_i$ because Constraint (1) ensures each task $i$ starts exactly once. The objective could equivalently be written as:
\[
\text{Maximize} \quad Z = \alpha \sum_{s \in T} L_s - \beta \sum_{i \in I} (p_i \cdot d_i) \left( \sum_{s \in T} X_{i,s} \right)
\]

\section{Constraints}

\subsection{(1) Task Assignment}
Each task must be assigned to start in exactly one time slot.
\[
\sum_{s \in T} X_{i,s} = 1 \quad \forall i \in I
\]

\subsection{(2) Deadline Constraint}
A task $i$ cannot start at slot $s$ if it would finish after its deadline $dl_i$. The task occupies slots $s, s+1, \dots, s + dur_i - 1$. The last occupied slot must be $\le dl_i$.
\[
X_{i,s} = 0 \quad \forall i \in I, s \in T \text{ such that } s + dur_i - 1 > dl_i
\]

\subsection{(3) No Overlap Constraint}
At most one task can occupy any given time slot $t$. A task $i$ starting at $st$ occupies slot $t$ if $st \le t < st + dur_i$.
\[
\sum_{i \in I} \sum_{\substack{st \in T \\ st \le t < st + dur_i}} X_{i, st} \le 1 \quad \forall t \in T
\]

\subsection{(4) Daily Scheduling Window Constraint}
Tasks must start and end within the specified daily time window ($h_{start}$ to $h_{end}$).
\begin{itemize}
    \item Tasks cannot start before the window begins: Let $s_{day} = s \pmod{\text{SLOTS\_PER\_DAY}}$.
    \[
    X_{i,s} = 0 \quad \forall i \in I, s \in T \text{ such that } s_{day} < s_{offset}^{start}
    \]
    \item Tasks cannot start so late that they end after the window closes:
    \[
    X_{i,s} = 0 \quad \forall i \in I, s \in T \text{ such that } s_{day} \ge s_{offset}^{end} - dur_i + 1
    \]
    (This ensures that the last slot $s + dur_i - 1$ has an index modulo SLOTS\_PER\_DAY less than $s_{offset}^{end}$).
\end{itemize}

\subsection{(5) Time Preference Constraint}
A task $i$ can only start in a slot $s$ that belongs to its preferred set $Pref_i$.
\[
X_{i,s} = 0 \quad \forall i \in I, s \in T \text{ such that } s \notin Pref_i
\]

\subsection{(6) Blocked Time / Commitment Constraint}
A task $i$ cannot start at slot $st$ if any of the slots it would occupy ($st, \dots, st + dur_i - 1$) are blocked ($commit_t = 15$).
\[
X_{i,st} = 0 \quad \forall i \in I, st \in T, \text{ if there exists } t \in \{st, \dots, st + dur_i - 1\} \text{ such that } commit_t = 15
\]
Alternatively, formulated slot by slot: For any blocked slot $t$ (where $commit_t=15$), no task $i$ can be active during that slot.
\[
\sum_{i \in I} \sum_{\substack{st \in T \\ st \le t < st + dur_i}} X_{i, st} = 0 \quad \forall t \in T \text{ such that } commit_t = 15
\]
(Note: The Python code implements this by directly setting $X_{i,st}=0$ if the task starting at $st$ overlaps with *any* blocked slot $t$.)

\subsection{(7) Leisure Calculation Constraints}
These constraints link the task assignments ($X_{i,s}$) to the slot occupancy ($Y_s$) and the leisure time ($L_s$).
\begin{itemize}
    \item Link $X$ to $Y$: If any task occupies slot $s$, $Y_s$ must be 1. Due to the non-overlap constraint (3), the sum is always $\le 1$.
    \[
    \sum_{i \in I} \sum_{\substack{st \in T \\ st \le s < st + dur_i}} X_{i, st} \le Y_s \quad \forall s \in T
    \]
    \item Link $Y$ and $commit$ to $L$: Leisure time $L_s$ in a slot $s$ is at most the total time in the slot (15 minutes) minus any committed time, and is only non-zero if the slot is not occupied by a task ($Y_s=0$).
    \[
    L_s \le (15 - commit_s) \cdot (1 - Y_s) \quad \forall s \in T
    \]
    Since $L_s \ge 0$ is also enforced, $L_s$ will be exactly $(15 - commit_s)$ if $Y_s=0$ and $commit_s=0$, exactly $0$ if $Y_s=1$, and exactly $0$ if $commit_s=15$. The objective function maximizing $L_s$ ensures $L_s$ takes its maximum possible value allowed by this constraint.
\end{itemize}

\subsection{(8) Variable Types}
Ensure variables adhere to their defined types.
\begin{align*}
X_{i,s} &\in \{0, 1\} \quad \forall i \in I, s \in T \\
Y_s &\in \{0, 1\} \quad \forall s \in T \\
L_s &\ge 0 \quad \forall s \in T
\end{align*}

\end{document}
