\documentclass{article}
\usepackage{amsmath} % For mathematical equations
\usepackage{geometry} % For adjusting margins
\usepackage{amssymb} % For sets (\mathbb{R}, etc. if needed, and \emptyset)
\usepackage{amsthm} % For definition environment if needed

% Adjust page margins
\geometry{a4paper, margin=1in}

% Improve paragraph spacing
\setlength{\parskip}{0.5em}
\setlength{\parindent}{0em}

\title{Mathematical Optimization Model for Weekly Task Scheduling}
\author{Formulation Documentation}
\date{\today}

\begin{document}

\maketitle

\section{Overview}

This document presents the mathematical formulation of an optimization model designed to generate a weekly task schedule. The model assigns start times to tasks within a defined time horizon, respecting various constraints, and optimizes a weighted combination of maximizing leisure time and minimizing task-related stress.

\subsection*{Time Horizon}
The planning horizon covers $D = 7$ days. Each day consists of $S_{day} = 56$ discrete time slots of 15 minutes each, typically representing the period from 8:00 AM to 10:00 PM. The total number of slots in the horizon is $S_{total} = D \times S_{day} = 392$. Slots are indexed globally from $s = 0$ to $S_{total}-1$.

\section{Parameters and Inputs}

The model utilizes the following input parameters:

\begin{itemize}
    \item \textbf{Set of Tasks ($T$):} The collection of tasks to be scheduled, indexed by $i$.
    \item \textbf{Task Attributes ($\forall i \in T$):}
    \begin{itemize}
        \item $p_i$: Numerical priority of task $i$.
        \item $d_i$: Numerical difficulty of task $i$.
        \item $dur_i$: Duration of task $i$, measured in the number of time slots.
        \item $dl_i$: Deadline for task $i$, represented as the global slot index by which the task must be fully completed.
        \item $AllowedSlots_i$: A subset of $\{0, ..., S_{total}-1\}$ indicating the permissible start slots for task $i$ based on its time preference (e.g., "morning", "afternoon").
    \end{itemize}
    \item \textbf{Set of Committed Slots ($C$):} A subset of $\{0, ..., S_{total}-1\}$ representing time slots that are pre-allocated and unavailable for scheduling tasks (e.g., meetings, appointments).
    \item \textbf{Objective Weights:}
    \begin{itemize}
        \item $\alpha \ge 0$: Weight coefficient emphasizing the maximization of leisure time.
        \item $\beta \ge 0$: Weight coefficient emphasizing the minimization of the total stress score.
    \end{itemize}
    \item \textbf{Daily Limit ($Limit_{daily}$, Optional):} An integer representing the maximum number of slots that can be occupied by tasks on any single day $d \in \{0, ..., D-1\}$.
\end{itemize}

\section{Decision Variables}

The model determines the values of the following variables:

\begin{enumerate}
    \item \textbf{$X_{i,s}$ (Binary):} Indicates if task $i$ starts at slot $s$.
    \[ X_{i,s} = \begin{cases} 1 & \text{if task } i \in T \text{ starts at slot } s \in \{0, ..., S_{total}-1\} \\ 0 & \text{otherwise} \end{cases} \]

    \item \textbf{$Z_{i}$ (Binary):} Indicates if task $i$ is scheduled at all.
    \[ Z_{i} = \begin{cases} 1 & \text{if task } i \in T \text{ is scheduled} \\ 0 & \text{otherwise} \end{cases} \]
    
    \item \textbf{$Y_{s}$ (Binary):} Indicates if slot $s$ is occupied by any task.
    \[ Y_{s} = \begin{cases} 1 & \text{if slot } s \in \{0, ..., S_{total}-1\} \text{ is occupied by a task} \\ 0 & \text{otherwise} \end{cases} \]
    This is an auxiliary variable linked to $X_{i,s}$ via constraints.

    \item \textbf{$L_{s}$ (Continuous):} Represents the leisure time (in minutes) available in slot $s$.
    \[ L_{s} \in [0, 15] \quad \forall s \in \{0, ..., S_{total}-1\} \]
\end{enumerate}

\section{Objective Function}

The objective is to maximize a weighted sum reflecting the trade-off between leisure and stress:

\[
\text{Maximize} \quad Z = \alpha \sum_{s=0}^{S_{total}-1} L_s - \beta \sum_{i \in T} \sum_{s=0}^{S_{total}-1} (p_i \times d_i) X_{i,s}
\]

\textbf{Explanation:}
\begin{itemize}
    \item The first term, $\alpha \sum L_s$, promotes maximizing the total leisure time across all slots. The weight $\alpha$ scales the importance of leisure relative to stress reduction.
    \item The second term, $\beta \sum (p_i \times d_i) X_{i,s}$, aims to minimize the total "stress" incurred. Stress for each task $i$ is defined as $p_i \times d_i$ and is counted if the task starts ($X_{i,s}=1$). The weight $\beta$ scales the penalty associated with scheduling high-priority or difficult tasks. This formulation associates stress with the initiation of a task, not its entire duration.
\end{itemize}

\section{Constraints}

The following constraints define the feasible schedules:

\subsection{Task Assignment and Selection}

Instead of requiring every task to be scheduled, the model now uses a target completion rate and allows selecting which tasks to schedule:

\[
Z_i = \sum_{s=0}^{S_{total}-1} X_{i,s} \quad \forall i \in T
\]

\textbf{Explanation:} This links the task selection variable $Z_i$ to the start slot variables $X_{i,s}$ for every task $i$. If task $i$ is scheduled to start at any slot ($X_{i,s}=1$ for some $s$), then $Z_i=1$. Otherwise, $Z_i=0$ (task is not scheduled).

\subsection{Target Completion Rate}
Ensures that a specified minimum percentage of tasks are scheduled:

\[
\sum_{i \in T} Z_{i} \geq \lceil |T| \times \text{target\_rate} \rceil
\]

\textbf{Explanation:} The constraint ensures that at least a target percentage (default: 70\%) of all tasks are scheduled. This provides a more realistic model as it doesn't assume all tasks must be completed. The ceiling function $\lceil \cdot \rceil$ ensures a whole number of tasks are scheduled.

\subsection{Hard Task Limitation}
Restricts the number of difficult tasks scheduled on any single day:

\[
\sum_{i \in T_{\text{hard}}} \sum_{s \in \text{Slots}_{day,d}} X_{i,s} \leq 1 \quad \forall d \in \{0, ..., D-1\}
\]

where $T_{\text{hard}} = \{i \in T : d_i \geq \text{hard\_threshold}\}$ is the subset of tasks with difficulty ratings at or above the specified threshold (default: 4).

\textbf{Explanation:} For each day $d$, this constraint sums the start variables $X_{i,s}$ for all hard tasks $i$ and all slots $s$ on day $d$. By limiting this sum to at most 1, the constraint ensures that at most one hard task can start on any given day. This prevents overloading days with too many challenging tasks.

\subsection{Deadlines and Horizon}
Ensures that tasks are completed by their deadline and fit entirely within the scheduling horizon.

\[
X_{i,s} = 0 \quad \forall i \in T, \forall s \text{ such that } (s + dur_i - 1 > dl_i) \lor (s + dur_i > S_{total})
\]

\textbf{Explanation:}
\begin{itemize}
    \item \textbf{Deadline Check ($s + dur_i - 1 > dl_i$):} A task $i$ starting at slot $s$ occupies slots from $s$ to $s + dur_i - 1$. The term $s + dur_i - 1$ represents the index of the last slot occupied by the task. This constraint forces the start variable $X_{i,s}$ to be 0 if the last occupied slot exceeds the task's specified deadline slot $dl_i$. This prevents tasks from starting if they cannot finish on time.
    \item \textbf{Horizon Check ($s + dur_i > S_{total}$):} The term $s + dur_i$ represents the index of the slot immediately following the task's completion. If this index is greater than the total number of slots $S_{total}$, it means the task would extend beyond the planning horizon. This constraint forces $X_{i,s}$ to 0 in such cases, ensuring tasks are fully contained within the schedule.
\end{itemize}

\subsection{No Overlap}
Prevents two or more tasks from being active in the same time slot.

\[
\sum_{i \in T} \sum_{s = \max(0, t - dur_i + 1)}^{t} X_{i,s} \le 1 \quad \forall t \in \{0, ..., S_{total}-1\}
\]

\textbf{Explanation:} For any given time slot $t$, this constraint considers all tasks $i$. The inner sum iterates through potential start slots $s$ for task $i$. The range $\max(0, t - dur_i + 1)$ to $t$ identifies precisely those start slots $s$ such that task $i$, if started at $s$, would be active during slot $t$. By summing the $X_{i,s}$ variables for all such task/start-slot combinations, we count how many tasks are active in slot $t$. Constraining this sum to be less than or equal to 1 ensures that at most one task can occupy any single slot $t$.

\subsection{Preferences}
Restricts the starting time of tasks to their allowed time windows (e.g., morning, afternoon).

\[
X_{i,s} = 0 \quad \forall i \in T, \forall s \notin AllowedSlots_i
\]

\textbf{Explanation:} For each task $i$, $AllowedSlots_i$ is the set of permissible start slots based on user preference. This constraint directly enforces the preference by setting the start variable $X_{i,s}$ to 0 for any slot $s$ that is not within the task's allowed set. This prevents the task from starting outside its desired time window(s).

\subsection{Commitments}
Prevents any part of a scheduled task from coinciding with predefined fixed commitments.

\[
X_{i,s} = 0 \quad \forall i \in T, \forall s \text{ such that } \{s, s+1, ..., s + dur_i - 1\} \cap C \neq \emptyset
\]

\textbf{Explanation:} For each task $i$ and potential start slot $s$, the set $\{s, s+1, ..., s + dur_i - 1\}$ represents all slots the task would occupy if it started at $s$. This constraint checks if this set has any overlap (non-empty intersection, $\cap C \neq \emptyset$) with the set of committed slots $C$. If an overlap exists, it means starting task $i$ at $s$ would conflict with a commitment. Therefore, the constraint forces $X_{i,s}$ to 0, preventing the task from starting at that time.

\subsection{Leisure Calculation and Occupation Link (Y)}
Links the auxiliary occupation variable $Y_s$ to the task start variables $X_{i,s}$ and defines the leisure time $L_s$ based on occupation and commitments.

\begin{align}
Y_s &= \sum_{i \in T} \sum_{start = \max(0, s - dur_i + 1)}^{s} X_{i,start} \quad &&\forall s \in \{0, ..., S_{total}-1\} \label{eq:link_y} \\
L_s &= 0 \quad &&\forall s \in C \label{eq:leisure_commit} \\
L_s &\le 15 \times (1 - Y_s) \quad &&\forall s \notin C \label{eq:leisure_task} \\
L_s &\ge 0 \quad &&\forall s \label{eq:leisure_nonneg}
\end{align}

\textbf{Explanation:}
\begin{itemize}
    \item \textbf{Equation (\ref{eq:link_y}):} This constraint defines the occupation variable $Y_s$. It equates $Y_s$ to the same sum used in the "No Overlap" constraint. Since the No Overlap constraint ensures that this sum can only be 0 or 1, this equality correctly sets $Y_s = 1$ if slot $s$ is occupied by any task, and $Y_s = 0$ otherwise. This makes $Y_s$ a reliable binary indicator of task occupation for slot $s$.
    \item \textbf{Equation (\ref{eq:leisure_commit}):} If slot $s$ belongs to the set of commitments $C$, no leisure time is possible, so $L_s$ is forced to 0.
    \item \textbf{Equation (\ref{eq:leisure_task}):} If slot $s$ is not committed ($s \notin C$), this constraint limits the leisure time. If the slot is occupied by a task ($Y_s = 1$), then $1 - Y_s = 0$, forcing $L_s \le 0$. Since $L_s$ must also be non-negative (Eq. \ref{eq:leisure_nonneg}), $L_s$ becomes exactly 0. If the slot is not occupied by a task ($Y_s = 0$), then $1 - Y_s = 1$, allowing $L_s \le 15$. The objective function, seeking to maximize $\sum L_s$, will push $L_s$ to its upper bound of 15 in this unoccupied, non-committed case.
    \item \textbf{Equation (\ref{eq:leisure_nonneg}):} Explicitly states (or is implied by variable definition) that leisure time cannot be negative.
\end{itemize}

\subsection{Daily Limits (Optional)}
Enforces a maximum total duration of tasks scheduled within any single day.

Let $Slots_{day, d} = \{s \mid d \times S_{day} \le s < (d+1) \times S_{day}\}$ be the set of slots for day $d$.
\[
\sum_{s \in Slots_{day, d}} Y_s \le Limit_{daily} \quad \forall d \in \{0, ..., D-1\}
\]

\textbf{Explanation:} This constraint operates on a per-day basis ($d=0$ to $D-1$). For each day, it sums the occupation variables $Y_s$ over all slots $s$ belonging to that day ($Slots_{day, d}$). Since $Y_s=1$ if a slot is occupied by a task and 0 otherwise, this sum represents the total number of task-occupied slots on day $d$. The constraint ensures this total does not exceed the specified $Limit_{daily}$.

\section{Output}

If the optimization problem is feasible and a solution is found (potentially optimal or feasible within a time limit), the model output provides the values of the decision variables. This typically translates to:
\begin{itemize}
    \item The optimal objective function value $Z$.
    \item A schedule detailing the assigned start slot $s$ (where $X_{i,s}=1$) for each selected task $i$ (where $Z_i=1$).
    \item The calculated total leisure time $\sum L_s$.
    \item The calculated total stress score $\sum (p_i \times d_i) X_{i,s}$.
    \item The achieved task completion rate $\frac{\sum Z_i}{|T|}$.
\end{itemize}

\end{document}
